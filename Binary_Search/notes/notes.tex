\documentclass{article}
\usepackage{xcolor}
\usepackage{hyperref}
\usepackage{amsmath}
\usepackage{geometry}
\usepackage{background}

\title{Notas de Búsqueda Binaria}
\author{Jorge Tapia}
\date{\today}

% Configuración de los márgenes de la página
\geometry{
    left=2cm,
    right=2cm,
    top=2cm,
    bottom=2cm,
}

% Definir el color del borde
\definecolor{bordercolor}{RGB}{0,0,0}

% Configuración del borde
\backgroundsetup{
    scale=1,
    color=black,
    opacity=1,
    angle=0,
    pages=some,
    contents={
        \begin{tikzpicture}[remember picture,overlay]
            \draw [line width=3pt,color=bordercolor]
                ($ (current page.north west) + (1.5cm,-1.5cm) $)
                rectangle
                ($ (current page.south east) + (-1.5cm,1.5cm) $);
        \end{tikzpicture}
    }
}

\begin{document}
\maketitle
\section{Interesting drink \href{https://codeforces.com/contest/705/problem/B}{\textcolor{blue}{(Link)}}}
\textbf{Etiquetas}: Binary search \\
En resumen, en este problema solo se trata de ordenar en orden no decreciente y encontrar el menor número que sea mayor o 
igual a cada consulta \( q_i \), ya que esa será la cantidad de bebidas que podrá tomar.
En otras palabras, solo necesitas hacer un \textbf{Upperbound} para cada consulta \( q_i \).
\section{Maximum Median \href{https://codeforces.com/contest/1201/problem/C}{\textcolor{blue}{(Link)}}}
\textbf{Etiquetas}: Binary search \\
Este problema consiste en hallar el numero máximo del la mediana aumentado \(a_i + 1\) hasta máximo $k$ operaciones, entonces lo que hay que hacer es no pasarse del rango $k$ haciendo la siguiente sumatoria a cada paso de la búsqueda binaria en otras palabras:
    \begin{equation*}
        \sum_{i=(n+1)/2}^{n} max(0,x-b_i)
    \end{equation*}
    
\end{document}

